\capitulo{2}{Objetivos del proyecto}

El propósito del proyecto es realizar una interfaz gráfica que permita controlar y monitorizar en tiempo real el funcionamiento de una o varias impresoras 3D.

Además se pretende definir diferentes niveles/perfiles de usuario, cada uno con sus permisos y acciones disponibles de manera que se pueda llevar a cabo un control y una gestión de los recursos de forma segura.

\section{Objetivos funcionales}
\begin{itemize}
\tightlist
\item Monitorizar el estado de una granja de impresoras 3D conectadas entre sí.
\item
    Crear un sistema de gestión de usuarios y permisos para garantizar la seguridad de la aplicación.
\item
    Permitir comenzar la impresión de una pieza si previamente hemos pre-cargado un G-code \cite{wiki:gcode}.
\item
    Permitir pausar una impresión cuando esté en curso.
\item
    Permitir cancelar una impresión cuando esté en curso.
\item
    Permitir conectar una impresora 3D desde la propia aplicación.
\item
    Permitir desconectar una impresora 3D desde la propia aplicación.

    
\end{itemize}

\clearpage

\section{Objetivos no funcionales}
\begin{itemize}
\tightlist
\item
    La aplicación se adaptará dinámicamente a cualquier resolución de pantalla y a cualquier tipo de dispositivo.
\item
	La aplicación será lo suficientemente intuitiva para que cualquier usuario sea capaz de utilizarla sin necesidad de ningún tipo de cualificación técnica.
\item
    La aplicación deberá recargarse cada pocos segundos con el fin de que la información sea lo más fiable posible.
\item
	La aplicación utilizará únicamente la red local para las comunicaciones sin necesitar conexión a Internet.

\end{itemize}