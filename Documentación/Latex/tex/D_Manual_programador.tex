\chapter[Documentación de programación]{Documentación técnica de programación}


\section{Introducción}

En este apartado de los anexos se explican los conceptos relacionados con el manual del programador y con la compilación, instalación y ejecución del proyecto para que alguna persona pueda continuar con el desarrollo de la aplicación web.

\section{Estructura de directorios}

El desarrollo de nuestro proyecto se ha llevado a cabo utilizando los siguientes directorios principales:

\begin{itemize}
\item \textbf{FlaskApp}: contiene el código fuente del proyecto con todos los archivos necesarios para ejecutar la aplicación.
\item \textbf{Documentación}: contiene la documentación para poder comprender correctamente el funcionamiento de nuestra aplicación. 
\end{itemize}

\imagen{estructura}{Estructura de los directorios del proyecto.}
\section{Manual del programador}

\subsection{Instalación de Anaconda}

La forma más sencilla de instalar un intérprete de 	\textit{Python} es instalando Anaconda. Para ello nos dirigimos a su web \cite{anaconda} y deberemos pulsar sobre el botón \textit{Download} tal y como se ve en la siguiente figura:

\imagen{anaconda1}{Descarga del intérprete de Python.}

Una vez hayamos descargado el archivo deberemos seguir las introducciones del instalador que aparece en pantalla.

\imagen{anaconda2}{Instalador de Anaconda.}
\subsection{Instalación de PyCharm}

PyCharm el \textit{IDE} que hemos elegido para llevar a cabo la codificación de nuestro proyecto. Para instalar dicho entorno de desarrollo nos tenemos que dirigir a su página web \cite{pycharm} y pulsar sobre la versión que queremos descargar. Tal y como comentamos en la memoria gracias a la cuenta universitaria que nos otorga la Universidad de Burgos tenemos acceso a una licencia "Profesional" de este entorno de desarrollo, en nuestro caso es la versión que hemos utilizado.

\imagen{Pycharm}{Descarga del entorno de desarrollo \textit{PyCharm}.}

Cuando la descarga se haya completado deberemos seguir el instalador como tantas veces hemos hecho.

\subsection{Instalación de Sublime Text}

Sublime Text es un completo editor de texto que hemos utilizado sobre todo para editar los archivos \textit{HTML} que contienen las vistas de la aplicación. Para descargar éste editor de texto deberemos dirigirnos a su página web \cite{sublime} y pulsar sobre el botón de descargar.

\imagen{sublime}{Descarga del editor de texto \textit{Sublime Text}.}

Cuando se haya terminado de descargar seguiremos el instalador.

 
\subsection{Instalación de PuTTY}

PuTTY es el programa que hemos usado para gestionar, mediante el protocolo \textit{SSH}, el servidor donde está corriendo nuestra aplicación web. Para instalar éste programa debemos ir a su página web \cite{putty} y pulsar sobre el botón descargar.

\imagen{putty}{Descarga del programa 	\textit{PuTTY}.}

Una vez descargado seguiremos los pasos del instalador.

\section{Compilación, instalación y ejecución del proyecto}

Para instalar y ejecutar el proyecto debemos tener en cuenta que necesitamos tener al menos una instancia de OctoPrint instalada y funcionando sobre un servidor. 
Cuando hayamos cumplido esta condición deberemos seguir los siguientes pasos para la ejecución del proyecto:

\subsection{Importar el repositorio}

Lo primero que debemos hacer es dirigirnos a la página web \cite{repositorio} del repositorio y descargarnos el proyecto desde el botón verde que pone \textit{Clone or download}  y luego en \textit{download zip} tal y como se ve en la siguiente figura:

\imagen{clone}{Botón para descargar el repositorio.}

Una vez hemos descargado y descomprimido el proyecto nos debemos dirigir al entorno de desarrollo que estemos usando, en nuestro caso \textit{PyCharm} y pulsar sobre el botón \textit{Open}.

\imagen{open}{Botón para importar el proyecto.}

Lo siguiente que debemos hacer es elegir el proyecto que acabamos de descargar y pulsar otra vez sobre el botón \textit{Open}  tal y como vemos en la siguiente imagen:

\imagen{ruta}{Elegir la ruta donde se encuentra el proyecto.}

Si hemos seguido estos pasos correctamente se nos abrirá nuestro entorno de desarrollo con el proyecto que hemos descargado de \textit{GitHub}.

\subsection{Ejecución}

Para ejecutar nuestro proyecto deberemos editar los archivos CSV que se encuentran dentro de nuestro proyecto y tendremos que copiar el token de API que nos proporciona OctoPrint y pegarlo en los archivos CSV que hemos comentado. Además también tendremos que cambiar el nombre de la impresora con el que queremos que se muestre en nuestra plataforma.

Otra cosa que debemos modificar es la dirección IP sobre la que está corriendo el servidor: para ello nos dirigimos al archivo \textit{monitorOcto.py} y deberemos modificar la variable llamada \textit{host} tal y como vemos en la siguiente figura:

\imagen{host}{Dirección sobre la que corre el servidor.}


Una vez hayamos completado todos los pasos anteriores debemos posicionarnos sobre el archivo \textit{monitorOcto.py} pulsar con el botón derecho y hacer click sobre la opción \textit{run 'monitorOcto'} tal y como vemos en la siguiente figura:

\imagen{run}{Ejecutar el proyecto.}

Una vez esté el proyecto corriendo deberemos irnos a nuestro navegador predeterminado y escribir en la barra de busqueda \textit{localhost} y nos cargará una página con el \textit{login} de la aplicación que hemos desarrollado.

